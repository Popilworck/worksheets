\documentclass{article}
\usepackage{graphicx} % Required for inserting images
\usepackage{amsmath}
\usepackage{amssymb}
\usepackage{enumitem}
\usepackage{subfig}

\begin{document}
\begin{center}
\textbf{
{\Large HKN ECE 120 Midterm 1 Worksheet}
}
\end{center} 
\noindent\makebox[\linewidth]{\rule{\linewidth}{0.2pt}}


\section*{Binary Representations}
\subsection*{Problem 1}
Write these conversions in decimal. Truncate if necessary.
\begin{enumerate}[label=\alph*.]
    \item Convert $100101_2$ to a 6-bit unsigned integer.
    \item Convert $100101_2$ to a 6-bit signed magnitude integer.
    \item Convert $100101_2$ to a 6-bit 2's complement integer.
    \item Convert $0 1110 1110_2$ to a 9-bit unsigned integer.
    \item Convert $0 1110 1110_2$ to a 9-bit 2's complement integer.
    \item Convert $1001 0010 1101_2$ to a 11-bit unsigned integer.
    \item Convert $1001 0010 1101_2$ to a 9-bit 2's complement integer.
    \item Convert $0010 1101_2$ to a 12-bit unsigned integer.
    \item Convert $10111_2$ to a 16-bit signed integer. 
\end{enumerate}

\subsection*{Problem 2}
Write these conversions in binary. Truncate if necessary.
\begin{enumerate}[label=\alph*.]
    \item Convert $51_{10}$ to a 8-bit unsigned integer.
    \item Convert $51_{10}$ to a 8-bit signed magnitude integer.
    \item Convert $51_{10}$ to a 8-bit 2's complement integer.
    \item Convert $-240_{10}$ to a 9-bit unsigned integer.
    \item Convert $-240_{10}$ to a 9-bit 2's complement integer.
    \item Convert $1171_{10}$ to a 11-bit unsigned integer.
    \item Convert $1171_{10}$ to a 11-bit 2's complement integer.
    \item Convert $65_{10}$ to a 12-bit unsigned integer.
    \item Convert $-23309_{10}$ to a 16-bit signed integer. 
\end{enumerate}

\section*{Other Representations}
\subsection*{Problem 1}
Convert these binary values to hexadecimal.
\begin{enumerate}[label=\alph*.]
    \item $0010 1011 0101 0110$
    \item $1001 0100 1000 1111$
    \item $0011 1100 0001 0010$
    \item $1011 1110 1110 1111$
    \item $1111 0000 0000 1101$
\end{enumerate}

\subsection*{Problem 2}
Convert these hexadecimal values to binary.
\begin{enumerate}[label=\alph*.]
    \item x37A5
    \item x2009
    \item x1F06
    \item x2FFE
    \item xDEADBEEF
\end{enumerate}

\subsection*{Problem 3}
Convert these hexadecimal values to ASCII.
\begin{enumerate}[label=\alph*.]
    \item x4A
    \item x2F
    \item x0D
    \item x4045
    \item x6E6F
\end{enumerate}

\subsection*{Problem 4}
Convert these ASCII characters to binary.
\begin{enumerate}[label=\alph*.]
    \item 'h'
    \item '#'
    \item 'M'
    \item '!'
    \item "bob"
\end{enumerate}

\subsection*{Problem 5}
True or False?
\begin{enumerate}[label=\alph*.]
    \item An integer with 11 hexadecimal values is at most a 88-bit integer.
    \item The shortest hexadecimal string that we can encode any 69-bit unsigned integer into is 18 characters long.
    \item All uppercase letters in ASCII start with the binary string $0100$.
    \item All lowercase letters in ASCII start with the binary string $011$.
    \item There is an ASCII character that corresponds with x8A.
    \item ASCII characters are usually stored as signed 8-bit integers.
    \item The control characters in ASCII were originally used as special codes for teletypes, keyboards used for electrical telegraphs. 
\end{enumerate}

\newpage
\section*{Binary Operations} % include bitmasks

\subsection*{Problem 1}
Perform the following operations. 
\begin{enumerate}[label=\alph*.]
    \item $1_2$ AND $0_2$
    \item $1_2$ OR $0_2$
    \item $10010010_2$ AND $01111011_2$
    \item $001010_2$ OR $111101_2$
    \item x8618 AND x7507
    \item $1_2$ XOR $1_2$
    \item xCA09 XOR x0990
    \item NOT $1001110100110101_2$
    \item $1001001101_2$ NAND $0110101110_2$
    \item $100011_2$ NOR $001000_2$
    \item x908 NXOR xA51
\end{enumerate}

\subsection*{Problem 2}
Perform the following operations on unsigned integers. Assume the number of bits given. Indicate when there is an overflow.
\begin{enumerate}[label=\alph*.]
    \item $100100_2 + 010101_2$
    \item $11011010_2 - 011010110_2$
    \item $1001_2 - 1010_2$
    \item $011101_2 + 111011_2$
    \item $1111000_2 \ll 2$ 
    \item $1111000_2 \gg 2$
    \item $000100_2 \gg 2$
\end{enumerate}

\subsection*{Problem 3}
Perform the following operations on signed integers. Assume the number of bits given. Indicate when there is an overflow.
\begin{enumerate}[label=\alph*.]
    \item $110010_2 + 110001_2$
    \item $11011010_2 + 011010110_2$
    \item $1001_2 - 1010_2$
    \item $011101_2 - 111011_2$
    \item $1111000_2 \ll 2$ 
    \item $1111000_2 \gg 2$
    \item $000100_2 \gg 2$
\end{enumerate}

\subsection*{Problem 4}
Bitmasks.
\begin{enumerate}[label=\alph*.]
    \item Suppose you have a 6-bit unsigned integer. What does applying AND $110000_2$ return? What does it indicate? 
    \item Suppose you have a 8-bit signed integer. What does applying AND $10000000_2$ return? What does it indicate? \\ \\ \\
    Suppose you have a 6-bit unsigned integer that represents 6 lights (1 = on, 0 = off).
    \item What operation and what mask should we use to enable a single light?
    \item What operation and what mask should we use to disable a single light?
    \item What operation and what mask should we use to toggle a single light?
    \item What operation can we use on these masks to form a new mask if we wanted to toggle more than one light?
\end{enumerate}

\newpage
\section*{K-maps and Optimization} % include area, delay heuristic
\subsection*{Problem 1}
Find the minimal SOP and POS expressions for the following table.
\begin{table}[!h]
\begin{tabular}{|l|l|l|l|}
\hline
A & B & C & S \\ \hline
0 & 0 & 0 & 1 \\ \hline
0 & 0 & 1 & 1 \\ \hline
0 & 1 & 0 & 0 \\ \hline
0 & 1 & 1 & 0 \\ \hline
1 & 0 & 0 & 0 \\ \hline
1 & 0 & 1 & 1 \\ \hline
1 & 1 & 0 & 1 \\ \hline
1 & 1 & 1 & 1 \\ \hline
\end{tabular}
\end{table}

\subsection*{Problem 2}
Find the minimal SOP and POS expressions for the following table.
\begin{table}[!h]
\begin{tabular}{|l|l|l|l|}
\hline
A & B & C & S \\ \hline
0 & 0 & 0 & 0 \\ \hline
0 & 0 & 1 & 1 \\ \hline
0 & 1 & 0 & X \\ \hline
0 & 1 & 1 & X \\ \hline
1 & 0 & 0 & 1 \\ \hline
1 & 0 & 1 & 0 \\ \hline
1 & 1 & 0 & X \\ \hline
1 & 1 & 1 & X \\ \hline
\end{tabular}
\end{table}

\newpage
\subsection*{Problem 3}
Find the minimal SOP and POS expressions for the following table.
\begin{table}[!h]
\begin{tabular}{|l|l|l|l|l|}
\hline
A & B & C & D & S \\ \hline
0 & 0 & 0 & 0 & 0 \\ \hline
0 & 0 & 0 & 1 & 1 \\ \hline
0 & 0 & 1 & 0 & 1 \\ \hline
0 & 0 & 1 & 1 & 0 \\ \hline
0 & 1 & 0 & 0 & 1 \\ \hline
0 & 1 & 0 & 1 & 1 \\ \hline
0 & 1 & 1 & 0 & 0 \\ \hline
0 & 1 & 1 & 1 & 1 \\ \hline
1 & 0 & 0 & 0 & 1 \\ \hline
1 & 0 & 0 & 1 & 0 \\ \hline
1 & 0 & 1 & 0 & 0 \\ \hline
1 & 0 & 1 & 1 & 0 \\ \hline
1 & 1 & 0 & 0 & 0 \\ \hline
1 & 1 & 0 & 1 & 0 \\ \hline
1 & 1 & 1 & 0 & 0 \\ \hline
1 & 1 & 1 & 1 & 0 \\ \hline
\end{tabular}
\end{table}

\subsection*{Problem 4}
Find the minimal SOP and POS expressions for the following table.
\begin{table}[!h]
\begin{tabular}{|l|l|l|l|l|}
\hline
A & B & C & D & S \\ \hline
0 & 0 & 0 & 0 & X \\ \hline
0 & 0 & 0 & 1 & 1 \\ \hline
0 & 0 & 1 & 0 & 0 \\ \hline
0 & 0 & 1 & 1 & 0 \\ \hline
0 & 1 & 0 & 0 & X \\ \hline
0 & 1 & 0 & 1 & X \\ \hline
0 & 1 & 1 & 0 & 1 \\ \hline
0 & 1 & 1 & 1 & 1 \\ \hline
1 & 0 & 0 & 0 & 1 \\ \hline
1 & 0 & 0 & 1 & 0 \\ \hline
1 & 0 & 1 & 0 & 0 \\ \hline
1 & 0 & 1 & 1 & 0 \\ \hline
1 & 1 & 0 & 0 & X \\ \hline
1 & 1 & 0 & 1 & X \\ \hline
1 & 1 & 1 & 0 & X \\ \hline
1 & 1 & 1 & 1 & X \\ \hline
\end{tabular}
\end{table}
\newpage 
\subsection*{Problem 5}
Find the area and delay heuristics for the following expressions. Do not include NOT gates.  
\begin{enumerate}[label=\alph*.]
    \item $ABC + A'B'C + C'$
    \item $A + B + C + D(A + B)$
    \item $ABCDEFGHIJKLMNOPQRSTUVWXY + Z$
    \item $(AB)'(A+B)'(CD)$
\end{enumerate}

\subsection*{Problem 6}
Implement the following expressions using AND and OR gates, then using NAND and NOR gates only. 
\begin{enumerate}[label=\alph*.]
    \item $AB + C$
    \item $A'B + AB' + ABC' + ABD'$
    \item $(A+B+C')(A'+B+C)(A+B'+C)$
    \item $(A+D)(B'+C'+A)$
\end{enumerate}


\newpage
\section*{IEEE 754 Floating Point}
\subsection*{Problem 1}
\begin{enumerate}[label=\alph*.]
    \item item 1
\end{enumerate}

\section*{C Basics} 
\subsection*{Problem 1}
\begin{enumerate}[label=\alph*.]
    \item item 1
\end{enumerate}

\section*{C Programming}
\subsection*{Problem 1}
\begin{enumerate}[label=\alph*.]
    \item item 1
\end{enumerate}


\end{document}

